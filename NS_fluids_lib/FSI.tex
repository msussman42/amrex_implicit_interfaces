\documentclass[]{article}

\par\noindent
A numerical method for simulating multimaterial compressible flows 
including the general interaction of fluids and elastic
material(s)
\par\noindent
Authors: Cody Estebe, Alireza Moradikazerouni, Mark Sussman, Kourosh Shoele
\par\noindent
Time step constraint:
\begin{description}
\item[CFL] $\bmu \Delta t < \Delta x/6$ (see Weymouth and Yue, EILE method of
Aulisa et al)
\item[Surface tension]
\item[elastic]
\end{description}

% https://ipe.otfried.org/
%see FSI_extend_cells
Given $n$ elastic materials, the algorithm is as follows:
\begin{description}
\item[1a. Refined advection] $S_{t}+\nabla\cdot(\bmu S)=F$
\item[1b. grid location project ] project velocity from refined grid
  to MAC grid.  Project volume fractions, levelset functions,
  temperature back to cell centers. Density and configuration tensor
  data stay on refined grid.
\item[1c. particle advection] directionally split particle advection
\item[1d. levelset redistancing] levelset redistancing
\item[2a. phase change advection] e.g. 
  $\dot{m}=[k\nabla T\cdot\nabla\bmn]/L$ where $\bmn$ points into the 
  material being consumed.
\item[2b. phase change particles] unsplit particle advection.
\item[2c. phase change redistancing] level set phase change redistancing
\item[3. volume preserving  phase change expansion source term] 
  $\dot{m}=|\delta F|$.
\item[4a. upper convective derivative or Jaumann derivative source terms]
  $S_{t}=\nabla\bmu S+S\nabla\bmu$ (elastic materials $1\le i\le n$)
\item[4b. extrapolate configuration tensor data] Extrapolate two
 grid cells.  $1\le i\le n$.
\item[5. save the current state (velocity field, interface(s), \ldots)]
  $\bmu,T,\rho,F,\phi$.
\item[6. coupling loop] For $i=1,n+1$
\item[7. restore state(s)] Using the saved states from step 5, restore
  the states of all materials except for the elastic materials $j=1,i-1$
\item[8. velocity extension] For $j=i,n$ extend the $j$th elastic material
 velocity in a band of thickness $\alpha \Delta x$.  $\alpha=1/2$.
\item[9. shift MAC grid density and viscosity] For $j=i,n$ shift the
  MAC density and viscosity $\beta \Delta x$ ($\beta=1/2$) 
  distance out of the $j$th elastic material.
\item[10. velocity diffusion] $\bmu_{t}=\nabla\cdot(2\mu D)/\rho$
\item[11a. grid location project configuration tensors $j=i,n$] 
  project tensor data
  from refined grid locations to strategic tensor grid locations
  (see Sugiyama et al). Note: the tensor data still lives on the refined
  grid, but projected to be strategically piecewise constant.
  In 2d, diagonal terms are at cell centers, and off diagonal are at nodes.
  In 3d, off diagonal are at edge centers.
\item[11b. Elastic force $j=i,n$] For $j=i,n$, include the elastic force
  $\nabla\cdot H G Q/rho$ where $G$ is the bulk modulus, $H$ is a shifted
  Heaviside function with shift length $\Delta x$ and thickness $\Delta x$.
  $\rho$ is the corresponding density for elastic material $j$.
\item[12. Compressible Pressure projection] The elastic materials, 
  $j=1,i-1$ are fixed (like a rigid solid).
\item[13. Go back to step 6 if $i<n+1$] Go back to step 6 if $i<n+1$.
\end{description}
%%
%%\bibliography{elasticpaper}
\end{document}
