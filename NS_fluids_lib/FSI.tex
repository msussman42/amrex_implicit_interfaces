\documentclass[]{article}

\par\noindent
A numerical method for simulating multimaterial compressible flows 
including the general interaction of fluids and elastic
material(s)
\par\noindent
Authors: Cody Estebe, Alireza Moradikazerouni, Mark Sussman, Kourosh Shoele
\par\noindent
Statement of the problem:
(see Jemison, Sussman, Arienti or Wenzel and Arienti)
\par\noindent
Let $\Omega_{m}(t)$ be the region occupied by material $m$ at time $t$.
\par\noindent
Define the indicator function:
\par\noindent
\begin{eqnarray}
	\xi_{m}(t,\bmx)=\left\{ \begin{array}{cc}
		1 & \bmx\in \Omega_{m}(t) \\
	0 & \mbox{otherwise}  \end{array} \right.
\end{eqnarray}
The volume fraction of material $m$ within a region $\Omega$ is
\begin{eqnarray}
	F_{m}=\frac{1}{|\Omega|}\int_{\Omega} \xi_{m}d\Omega
\end{eqnarray}
Define the state variable $\bmU$ as
\begin{eqnarray}
	\bmU=\left( \begin{array}{c}
		\xi_{m} \\ \rho_{m}\xi_{m} \\ \rho_{m}\xi_{m}\bmu_{m} \\
		\rho_{m}Y_{m,s}\xi_{m} \\
	\rho_{m}E_{m}\xi_{m} \end{array} \right)
\end{eqnarray}
Define the ``multidimensional flux vector'' as:
\begin{eqnarray}
	\bmF=\left( \begin{array}{c}
	\xi_{m}\bmu_{m} \\
	\rho_{m}\xi_{m}\bmu_{m} \\
	\bmu_{m}\ocirc\rho_{m}\xi_{m}\bmu_{m}+p_{m}I-\tau_{m} \\
	(\rho_{m}Y_{m,s}\bmu_{m}-\rho_{m}{\cal D}_{m,s}\nabla Y_{m,s})\xi_{m} \\
		(\rho_{m}E_{m}\bmu_{m}+p_{m}\bmu_{m}-\bmu_{m}\cdot \tau_{m}-
		k_{m}\nabla T_{m}+\bmq_{h,m})\xi_{m}
	\end{array} \right)
\end{eqnarray}


Time step constraint:
\begin{description}
\item[CFL] $\bmu \Delta t < \Delta x/6$ (see Weymouth and Yue, EILE method of
Aulisa et al)
\item[Surface tension]
\item[elastic]
\end{description}

% https://ipe.otfried.org/
%see FSI_extend_cells, H_radius (fort_elastic_force),
%related research: Sebastien Aland, Sugiyaman, Seungwon Shin
%A 3D phase-field based Eulerian variational framework for
%multiphase fluid–structure interaction with contact dynamics
%Xiaoyu Mao, Biswajeet Rath, Rajeev Jaiman 
%CMAME 2024
Given $n$ elastic materials, the algorithm is as follows:
\begin{description}
\item[1a. Refined advection] $S_{t}+\nabla\cdot(\bmu S)=F$
\item[1b. grid location project ] project velocity from refined grid
  to MAC grid.  Project volume fractions, levelset functions,
  temperature back to cell centers. Density and configuration tensor
  data stay on refined grid.
\item[1c. particle advection] directionally split particle advection
\item[1d. levelset redistancing] levelset redistancing
\item[2a. phase change advection] e.g. 
  $\dot{m}=[k\nabla T\cdot\nabla\bmn]/L$ where $\bmn$ points into the 
  material being consumed.
\item[2b. phase change particles] unsplit particle advection.
\item[2c. phase change redistancing] level set phase change redistancing
\item[3. volume preserving  phase change expansion source term] 
  $\dot{m}=|\delta F|$.
\item[4a. upper convective derivative or Jaumann derivative source terms]
  $S_{t}=\nabla\bmu S+S\nabla\bmu$ (elastic materials $1\le i\le n$)
\item[4b. extrapolate configuration tensor data] Extrapolate 
 \verb=ngrow_distance=
 grid cells.  $1\le i\le n$.  See \verb=fort_extrapolate_tensor=
\item[5. save the current state (velocity field, interface(s), \ldots)]
  $\bmu,T,\rho,F,\phi$.
\item[6. coupling loop] For $i=1,n+1$
\item[7. restore state(s)] Using the saved states from step 5, restore
  the states of all materials except for the elastic materials $j=1,i-1$
\item[8. velocity extension] For $j=i,n$ extend the $j$th elastic material
 velocity in a band of thickness $\alpha \Delta x$.  $\alpha=1$ if elastic
 and $\alpha=0$ if viscoelastic (i.e. velocity extension is disable for 
 viscoelastic materials).
\item[9. velocity diffusion] $\bmu_{t}=\nabla\cdot(2\mu D)/\rho$
\item[10a. grid location project configuration tensors $j=i,n$] 
  project tensor data
  from refined grid locations to strategic tensor grid locations
  (see Sugiyama et al). Note: the tensor data still lives on the refined
  grid, but projected to be strategically piecewise constant.
  In 2d, diagonal terms are at cell centers, and off diagonal are at nodes.
  In 3d, off diagonal are at edge centers.
\item[10b. Elastic force $j=i,n$] For $j=i,n$, include the elastic force
  $\nabla\cdot H G Q/rho$ where $G$ is the shear modulus, $H$ is a 
  Heaviside function with thickness $(ngrow_make_distance)\Delta x$.
  In the CMAME paper, Phase=0.97 if Levelset=-3 dx and Phase=-0.97 if 
  Levelset=3 dx.
  $\rho$ is the corresponding density for elastic material $j$.  
\item[11. Compressible Pressure projection] The elastic materials, 
  $j=1,i-1$ are fixed (like a rigid solid).
\item[12. Go back to step 6 if $i<n+1$] Go back to step 6 if $i<n+1$.
\end{description}
Distinctions between the present algorithm and Shin's algorithm: 
\begin{enumerate}
\item We do not multiply the configuration tensor by $H^{p}$ at $t=0$
  nor do we use $H^{1-p}$ in the elastic force term ($p=1/2$ in
  S. Shin's paper).  The rationale was that
  multiplying the configuration tensor by $H^{p}$ would make the 
  resulting configuration tensor field discontinuous, for which 
  numerical methods would continue to smear the discontinuity over time.
  Instead, we extrapolate the configuration tensor outward from the 
  elastic material 2 grid cells, every time step.
\item We impose a $\Delta x$ ``buffer zone'' around the elastic body
  when computing the updated elastic state.  
\item By design, since we use multimaterial nested dissection reconstruction
  and Weymouth and Yue advection, volume (for incompressible materials) is
  preserved to machine precision.
\end{enumerate}
Distinction between elastic material and viscoelastic materials:
\begin{enumerate}
\item In both cases, the staggered discretization for the configuration tensor
  is used; this discretization was introduced by Sugiyama et al.
\item $\alpha=\beta=\gamma=1$ for elastic materials, and 
  $\alpha=\beta=\gamma=0$ for viscoelastic materials. 
\item For elastic materials, there are separate pressure projection steps for
  establishing the state in elastic materials; the elastic material
  state is fixed for doing the ensuing ``fluids'' projection step.
  For viscoelastic materials, the state is established at the same time as
  for the other fluid materials.
\end{enumerate}
%%
%%\bibliography{elasticpaper}
\end{document}
